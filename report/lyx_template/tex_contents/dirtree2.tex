% This file was dirtree.tex. It was modified by TH. to support
% splitting the tree at pagebreaks sanely. It has been renamed to
% durtree.sty. It will now only work with LaTeX. It is meant as a
% quick hack to solve the problem at hand. Any bugs in it are my own,
% please don't blame M. Charpentier for them!
%
%%
%% This is file `dirtree.tex',
%% generated with the docstrip utility.
%%
%% The original source files were:
%%
%% dirtree.dtx  (with options: `tex')
%% 
%% IMPORTANT NOTICE:
%% 
%% For the copyright see the source file.
%% 
%% Any modified versions of this file must be renamed
%% with new filenames distinct from dirtree.tex.
%% 
%% For distribution of the original source see the terms
%% for copying and modification in the file dirtree.dtx.
%% 
%% This generated file may be distributed as long as the
%% original source files, as listed above, are part of the
%% same distribution. (The sources need not necessarily be
%% in the same archive or directory.)
%%
%% Package `dirtree.dtx'
%% -----------------------------------------------
%% Copyright (C) 2004-2006 Jean-C\^ome Charpentier
%% -----------------------------------------------
%%
%% This work may be distributed and/or modified under the
%% conditions of the LaTeX Project Public License, either version 1.3
%% of this license or (at your option) any later version.
%% The latest version of this license is in
%%   http://www.latex-project.org/lppl.txt
%% and version 1.3 or later is part of all distributions of LaTeX
%% version 2003/12/01 or later.
%%
%% See CTAN archives in directory macros/latex/base/lppl.txt.
%%
%% CONTENTS:
%%   This work consists of the files dirtree.ins and dirtree.dtx.
%%   Derived files are dirtree.tex and dirtree.sty.
%%
%% DESCRIPTION:
%%   dirtree is a package displaying directory trees.
%%
\def\fileversion{0.1}
\def\filedate{2010/12/04}
\NeedsTeXFormat{LaTeX2e}[1995/06/01]
\ProvidesPackage{durtree}[\filedate\space v\fileversion]
\RequirePackage{zref-abspage}
\long\def\LOOP#1\REPEAT{%
  \def\ITERATE{#1\relax\expandafter\ITERATE\fi}%
  \ITERATE
  \let\ITERATE\relax
}
\let\REPEAT=\fi
\def\@nameedef#1{\expandafter\edef\csname #1\endcsname}
\newdimen\DT@offset \DT@offset=0.2em
\newdimen\DT@width \DT@width=1em
\newdimen\DT@sep \DT@sep=0.2em
\newdimen\DT@all
\DT@all=\DT@offset
\advance\DT@all \DT@width
\advance\DT@all \DT@sep
\newdimen\DT@rulewidth \DT@rulewidth=0.4pt
\newdimen\DT@dotwidth \DT@dotwidth=1.6pt
\newdimen\DTbaselineskip \DTbaselineskip=\baselineskip
\newcount\DT@counti
\newcount\DT@countii
\newcount\DT@countiii
\newcount\DT@countiv
\newcount\DT@treenum \DT@treenum=0
\newif\ifDT@split
\def\DTsetlength#1#2#3#4#5{%
  \DT@offset=#1\relax
  \DT@width=#2\relax
  \DT@sep=#3\relax
  \DT@all=\DT@offset
  \advance\DT@all by\DT@width
  \advance\DT@all by\DT@sep
  \DT@rulewidth=#4\relax
  \DT@dotwidth=#5\relax
}
\def\DTstyle{\ttfamily}
\def\DTstylecomment{\rmfamily}
\def\DTcomment#1{%
  \kern\parindent\dotfill
  {\DTstylecomment{#1}}%
}
\def\dirtree#1{%
  \global\advance\DT@treenum by\@ne
  \let\DT@indent=\parindent
  \parindent=\z@
  \let\DT@parskip=\parskip
  \parskip=\z@
  \let\DT@baselineskip=\baselineskip
  \baselineskip=\DTbaselineskip
  \let\DT@strut=\strut
  \def\strut{\vrule width\z@ height0.7\baselineskip depth0.3\baselineskip}%
  \DT@counti=\z@
  \let\next\DT@readarg
  \next#1\@nil
  \dimen@=\hsize
  \advance\dimen@ -\DT@offset
  \advance\dimen@ -\DT@width
  \setbox\z@=\hbox to\dimen@{%
    \hsize=\dimen@
    \vbox{\@nameuse{DT@body@1}}%
  }%
  \dimen@=\ht\z@
  \advance\dimen@ by\dp\z@
  \advance\dimen@ by-0.7\baselineskip
  \ht\z@=0.7\baselineskip
  \dp\z@=\dimen@
  \par\leavevmode
  \kern\DT@offset
  \kern\DT@width
  \box\z@
  \endgraf
  \DT@countii=\@ne
  \DT@countiii=\z@
  \dimen3=\dimen@
  \@namedef{DT@lastlevel@1}{-0.7\baselineskip}%
  \loop
  \ifnum\DT@countii<\DT@counti
    \advance\DT@countii \@ne
    \advance\DT@countiii \@ne
    \dimen@=\@nameuse{DT@level@\the\DT@countii}\DT@all
    \advance\dimen@ by\DT@offset
    \advance\dimen@ by-\DT@all
    \leavevmode
    \kern\dimen@
    \DT@countiv=\DT@countii
    \count@=\z@
    \DT@splitfalse
    \LOOP
      \advance\DT@countiv \m@ne
      \ifnum\@nameuse{DT@level@\the\DT@countiv} >
        \@nameuse{DT@level@\the\DT@countii}\relax
      \else
        \count@=\@ne
      \fi
      \ifnum0\zref@extract{DT\the\DT@treenum.\the\DT@countiv}{abspage} =
        0\zref@extract{DT\the\DT@treenum.\the\DT@countii}{abspage}\relax
      \else
        \advance\DT@countiv\@ne
        \count@=\@ne
        \DT@splittrue
      \fi
    \ifnum\count@=\z@
    \REPEAT
    \edef\DT@hsize{\the\hsize}%
    \count@=\@nameuse{DT@level@\the\DT@countii}\relax
    \dimen@=\count@\DT@all
    \advance\hsize by-\dimen@
    \setbox\z@=\vbox{\@nameuse{DT@body@\the\DT@countii}}%
    \hsize=\DT@hsize
    \dimen@=\ht\z@
    \advance\dimen@ by\dp\z@
    \advance\dimen@ by-0.7\baselineskip
    \ht\z@=0.7\baselineskip
    \dp\z@=\dimen@
    \@nameedef{DT@lastlevel@\the\DT@countii}{\the\dimen3}%
    \advance\dimen3 by\dimen@
    \advance\dimen3 by0.7\baselineskip
    \dimen@=\@nameuse{DT@lastlevel@\the\DT@countii}\relax
    \advance\dimen@ by-\@nameuse{DT@lastlevel@\the\DT@countiv}\relax
    \advance\dimen@ by0.3\baselineskip
    \ifnum\@nameuse{DT@level@\the\DT@countiv} <
        \@nameuse{DT@level@\the\DT@countii}\relax
      \advance\dimen@ by-0.5\baselineskip
    \fi
    \ifDT@split
      \advance\dimen@ by.4\baselineskip
    \fi
    \kern-0.5\DT@rulewidth
    \hbox{\vbox to\z@{\vss\hrule width\DT@rulewidth height\dimen@}}%
    \kern-0.5\DT@rulewidth
    \kern-0.5\DT@dotwidth
    \vrule width\DT@dotwidth height0.5\DT@dotwidth depth0.5\DT@dotwidth
    \kern-0.5\DT@dotwidth
    \vrule width\DT@width height0.5\DT@rulewidth depth0.5\DT@rulewidth
    \kern\DT@sep
    \box\z@
    \endgraf
  \repeat
  \parindent=\DT@indent
  \parskip=\DT@parskip
  \DT@baselineskip=\baselineskip
  \let\strut\DT@strut
}
\def\DT@readarg.#1 #2. #3\@nil{%
  \advance\DT@counti \@ne
  \@namedef{DT@level@\the\DT@counti}{#1}%
  \edef\DT@label{DT\the\DT@treenum.\the\DT@counti}%
  \expandafter\def\csname DT@body@\the\DT@counti\expandafter\endcsname
      \expandafter{\expandafter\strut\expandafter\zref@label\expandafter{%
      \DT@label}%
    {\DTstyle{#2}\strut}%
  }
  \ifx\relax#3\relax
    \let\next\@gobble
  \fi
  \next#3\@nil
}
\endinput
%%
%% End of file `durtree.sty'.
