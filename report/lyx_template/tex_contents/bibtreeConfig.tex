%%%%%%%%%%%%%%%%%%%%%%%%%%%%%%%%%%%%%%%%%%%%%%%%%%%%%%%%%%%%%%%%%%%%%%%%%%%%%%%%
% This is the Bib-Tree config file. 
% If you want its an easy way to give an overview of the literature related
% to thr content of this thesis.
% As a roule of thumb there should be 5 to 10 references with about 2 to 4 
% levels of hierarchy mentioned in a single Bib-Tree.
%%%%%%%%%%%%%%%%%%%%%%%%%%%%%%%%%%%%%%%%%%%%%%%%%%%%%%%%%%%%%%%%%%%%%%%%%%%%%%%%
\chapter*{\BibTreeName}
\label{ch:bibtree}
\addcontentsline{toc}{chapter}{\BibTreeName}
\markboth{\BibTreeName}{\BibTreeName}
\vspace{1cm}

\newcommand{\comment}[1]{ \bfseries{{(#1)}}}

\newcommand{\negpar}[1][-1em]{%
  \ifvmode\else\par\fi
  {\parindent=#1\leavevmode}\ignorespaces
}

%h{offset}{width}{sep}{rule-width}{dot-size}
\DTsetlength{-1.15em}{0.5em}{0.8cm}{0.4pt}{1.6pt}
\setlength{\DTbaselineskip}{20pt}
% define a handy command to be used in the bib-tree
% one for references with author
\newcommand{\bibtree}[1] {\vspace{-0.6cm}\negpar[-0.80cm]\parbox[c]{1.50em}{\centering \cite{#1}} \usebibentry{#1}{author}, \usebibentry{#1}{title}, \usebibentry{#1}{year}.\vspace{0.2em}}%
% and one with missing author field
\newcommand{\bibtreeNa}[1] {\vspace{-0.6cm}\negpar[-0.80cm]\parbox[c]{1.5em}{\centering \cite{#1}} \usebibentry{#1}{title}, \usebibentry{#1}{year}.\vspace{0.2em}}%

\renewcommand*\DTstyle{}
% here follows the actual tree in respective LyX-File
